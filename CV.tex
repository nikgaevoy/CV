\documentclass[a4paper, 11pt]{article}

\usepackage{CV}

\usepackage{fancyhdr}
\pagestyle{fancy}

\fancyhead[L]{{Curriculum Vitae}}
\fancyhead[R]{{Updated~\today}}

\setlength{\headheight}{15pt}

\begin{document}

\nobibliography{biblio}
\bibliographystyle{splncs04}

\begin{center}
	{\Huge Nikita Gaevoy}
\end{center}

\begin{center}
	\begin{tabular}{lll}
		\textbf{Email}: \href{mailto:nikgaevoy@gmail.com}{{\it nikgaevoy@gmail.com}} &
		\hspace{0.13cm} \textbf{GitHub}: \href{https://github.com/nikgaevoy}{{\it @nikgaevoy}} &
		\hspace{0.13cm} \textbf{Phone}: \href{tel:+79219521012}{{\it +7\,921\,952\,10\,12}}
	\end{tabular}
\end{center}

\section*{Work Experience}

\begin{tabularx}{\textwidth}{lX}
	2021~-- Present & Junior Researcher at the Euler International
	Mathematical Institute.
	\\
	2018~-- 2021 & Junior Researcher at St. Petersburg Department of Steklov Mathematical Institute of Russian Academy of Sciences. Participant of the Russian Science Foundation grant no. 17-11-01276 ``Networking and distributed systems and algorithms and related fundamental problems'', head: S.\,I.\,Nikolenko.
	\\
\end{tabularx}


\section*{Education}

\begin{tabularx}{\textwidth}{lX}
	2021~-- Present & PhD in Mathematics at Saint Petersburg University, Russia, \href{https://math-cs.spbu.ru/en/advanced-mathematics/}{program ``Advanced Mathematics''} \\
	2019~-- 2021 & \href{https://diploma.spbu.ru/s/?rn=3121007\&bd=19980122\&h=67a15239b3294582867a44ba9e42cf98}{MSc in Mathematics} at Saint Petersburg University, Russia, \href{https://math-cs.spbu.ru/en/msc-math-en/}{program ``Advanced Mathematics''}, diploma with distinction \\
	2015~-- 2019 & \href{https://diploma.spbu.ru/s/?rn=0911007\&bd=19980122\&h=a34dc3393d004d5fb149a60e3545b673}{BSc in Mathematics} at Saint Petersburg University, Russia, \href{https://math-cs.spbu.ru/en/}{program ``Mathematics''}
\end{tabularx}

\subsubsection*{School Education}

\begin{tabularx}{\textwidth}{lX}
	2011~-- 2015 & \href{http://www.school30.spb.ru/}{Saint Petersburg Physics and Mathematics Lyceum \#30}
\end{tabularx}


\section*{Research interests}

SAT solvers, running time bounds for SAT algorithms, proof complexity, computational complexity, theoretical aspects of competitive programming (algorithms and data structures), algorithms for networking

\section*{Preprints}

\begin{itemize}
	\item \bibentry{Networking19}
	\item \bibentry{DPLL21} (based on Bachelor's thesis)
\end{itemize}

\section*{Master Thesis}

\begin{tabularx}{\textwidth}{lX}
	Title & Simulations between proof systems \\
	Supervisor & Prof.~Edward~A.~Hirsch \\
	Grade & Excellent
\end{tabularx}

\section*{Bachelor Thesis}

\begin{tabularx}{\textwidth}{lX}
	Title & The complexity of SAT algorithms \\
	Supervisor & Prof.~Edward~A.~Hirsch \\
	Grade & Excellent
\end{tabularx}

\section*{Teaching Experience}

\begin{tabularx}{\textwidth}{lX}
	2021~-- 2022 & Teaching assistant on the course ``Algorithms'' for Master's students at the National Research University ``Higher School of Economics'', St.Petersburg \\

	2021~-- 2022 & Teaching assistant on the course ``Algorithms'' for 2nd year Bachelor's students at the National Research University ``Higher School of Economics'', St.Petersburg  \\

	2020~-- 2022 & Teaching assistant on the course \href{https://users.math-cs.spbu.ru/~okhotin/teaching/algorithms_2020/}{``Mathematical foundations of algorithms''} for Bachelor's students of \href{https://math-cs.spbu.ru/en/}{program ``Mathematics''} at Saint Petersburg State University \\

	2017~-- Present & Jury of St.~Petersburg State University Cup \\

	2015~-- Present & Guest lecturer and teaching assistant at the Mathematics Club at Physics and Mathematics Lyceum \#30 \\

	2018~-- Present & Guest lecturer and teaching assistant at the Programming Club at Physics and Mathematics Lyceum \#30
\end{tabularx}

\section*{Awards and Achievements}

\begin{itemize}
	\item ICPC 2021 world finalist (as a member of SPb SU LOUD Enough team). The Finals has not happened yet.

	\item Google HashCode 2022 finalist, 16th place.

	\item \href{https://rucode.net/kak-eto-bylo/}{RuCode Festival}, April 2022, Champions (as a member of SPb SU LOUD Enough team).

	\item \href{https://rucode.net/kak-eto-bylo/}{RuCode Festival}, April 2021, 2nd place (as a member of SPb SU LOUD Enough team).

	\item \href{http://rucode.it-edu.mipt.ru/rucode2020resAB}{RuCode Festival}, April 2020, 2nd place (as a member of SPb SU LOUD Enough team).

	\item \href{https://codeforces.com/blog/entry/53192}{VK Cup 2017}, 5th place (jointly with Ivan~Bochkov).

	\item Three-time St.~Petersburg State University Champion (\href{https://acm.math.spbu.ru/cgi-bin/monitor.pl/n171015.dat}{XLVIII}, \href{https://acm.math.spbu.ru/cgi-bin/monitor.pl/n190421.dat}{LIII} and \href{https://acm.math.spbu.ru/cgi-bin/monitor.pl/n201206.dat}{LVI}).

	\item \href{https://math-cs.spbu.ru/en/scholarships-rodnye-goroda/}{Scholarship of ``Gazprom Neft''} prize winner in 2015, and then in 2016.

	\item Participant of the final stage of All-Russian Mathematical Olympiad in 2015.

	\item Participant of the final stage of All-Russian Programming Olympiad in 2015.

	\item Awardee of the final stage of All-Russian Team Programming Olympiad in 2015.

	\item Winner of School Olympiad of the Mathematics and Mechanics faculty of St.~Petersburg State University in 2015.
\end{itemize}

\section*{Competitive Programming}

\begin{tabular}{ll}
	Codeforces & \href{https://codeforces.com/profile/nikgaevoy}{{\it @nikgaevoy}} \\
	AtCoder & \href{https://atcoder.jp/users/nikgaevoy}{{\it @nikgaevoy}}
\end{tabular}

\section*{Contribution to Open Source}

\href{https://github.com/nikgaevoy/SPbTrueSkill}{Open-source Rust implementation} of the improved version of TrueSkill, which was used as base for other implementation for \href{https://arxiv.org/abs/2101.00400}{research projects}.

\section*{Programming Skills}

C++ (main language since 2013), Python, Java (+ Android), Rust.

\end{document}
